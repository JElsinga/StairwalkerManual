This section is going to discuss most of the possibilities that Geoserver has to offer when it comes to the visualization of the data. The GeoServer manual has some information on this subject, which can be found on: \url{http://docs.geoserver.org/2.5.x/en/user/styling/sld-reference/index.html}.

The following sections will give some more information on what the differences are between the different options and some ways to implement the different options. This section will use some of the examples used in our demo to give some more insight what can be done and hopefully making it easier to realise what is wanted. Most information can be found on the website, below will be a discussion on what types are best used when, but also some more information on how to make them work properly.

\subsubsection{Symbolizers}

In SLD there are three different symbolizers, a linesymbolizer, a pointsymbolizer and a polygonsymbolizer. A pointsymbolizer is used when the data that has to be represented is best shown as points. It does exactly what it says, you'll get a map with points on it and each point will represent a data-object from your dataset. This symbolizer can be really handy in certain situations. For example when you want to show all locations were a rare species of a bird has been found it will show a map with all points were a bird has been reported.

A linesymbolizer is used when the data that has to be displayed is best shown in lines. This symbolizer is best used to display roads for example. It isn't a symbolizer that can be used to represent data very well, but it is more used for pre-defined data, think off rivers, roads etc.

A polygonsymbolizer is used when the data that you want to represent has to be displayed in two-dimensional objects. There are many possibilities in a polygonsymbolizer. It is possible to make a simple square but it also has the option to make circles or triangles. It is one of the most commonly used symbolizers. It has the most potentional since you can do a lot with a polygonsymbolizer. A good example where a polygonsymbolizer is used, is to display the amount of people living in cities. This can be done with a circle polygon and that the circle will get bigger when more people live in a city.

\subsubsection{Filters}

Filters are the most important function when it comes to making a custom style. A filter is basically the basis of a fancy layer. What a filter does is that it makes a ruling and if that ruling is met, the color, labeling etc will be done. In SLD it is possible to have an unlimited amount of filters so the possibilities are endless. The following filter expression can be used:

\begin{itemize}
	\item \lstinline|$<$PropertyIsEqualTo$>$|
	\item \lstinline|$<$PropertyIsNotEqualTo$>$|
	\item \lstinline|$<$PropertyIsLessThan$>$|
	\item \lstinline|$<$PropertyIsLessThanOrEqualTo$>$|
	\item \lstinline|$<$PropertyIsGreaterThan$>$|
	\item \lstinline|$<$PropertyIsGreaterThanOrEqualTo$>$|
\end{itemize}
An example on how a single filter can be used is the following:
\begin{lstlisting}
<ogc:Filter>
  <ogc:PropertyIsLessThan>
    <ogc:PropertyName>testvalue</ogc:PropertyName>
    <ogc:Literal>2000000</ogc:Literal>
  </ogc:PropertyIsLessThan>
</ogc:Filter>
\end{lstlisting}
This example will test if the \lstinline|testvalue| is less than \lstinline|2000000|. If this is the case you can add what the filter should be doing. Below is the complete example that does something with this filters.
\begin{lstlisting}
<Rule>
  <Name>SmallPop</Name>
  <Title>Less Than 100</Title>
  <ogc:Filter>
    <ogc:PropertyIsLessThan>
      <ogc:PropertyName>testvalue</ogc:PropertyName>
      <ogc:Literal>100</ogc:Literal>
    </ogc:PropertyIsLessThan>
  </ogc:Filter>
  <PolygonSymbolizer>
    <Fill>
      <CssParameter name="fill">#38FF19
      </CssParameter>
      <CssParameter name="fill-opacity">1.0
      </CssParameter>
    </Fill>
    <Stroke>
      <CssParameter name="stroke">#000000
      </CssParameter>
      <CssParameter name="stroke-width">1.0
      </CssParameter>
    </Stroke>
  </PolygonSymbolizer>
</Rule>
\end{lstlisting}

What this example does is that if the \lstinline|testvalue| is below \lstinline|100|, it will fill a polygon with the color: \lstinline|#38FF19|. If this is not the case it will go to the next rule (if there is any otherwise it will just not do anything). The image shows a graph of the implementation we made for our data. The image has different kind of colors for the amount of data in a tile. If the amount is high the color will become more red and if there is little data the tile will be green. This is a good example on what can be done with filters.

\subsubsection{Extra Options}

GeoGerver SLD has a lot of options when it comes to customizing the data display that you've made. Below are some of the important features that are commonly used in Geoserver.

\textbf{Halo:} A halo gives a glow behind the current label. It should always be used in a textsymbolizer, since this is the only place you can add a halo. To use a halo its very simple, you do \lstinline|<Halo> </Halo>| and in between it is possible to add \lstinline|<Radius>| and \lstinline|<Fill>|. For more information on how to use a \lstinline|<Fill>| look in the GeoServer SLD cookbook.

\textbf{Anchorpoint:} An anchor point is a really handy tool to place your label on every place possible. It is used as shown below and important to notice is that you can set where the anchor point is (for example above the point) and you can displace it afterwards based on this anchor point, for instance make it go all the way to left (negative X placement) or all the way to the right (positive X placement)
\begin{lstlisting}
<PointPlacement>
  <AnchorPoint>
    <AnchorPointX>0.5</AnchorPointX>
    <AnchorPointY>0.0</AnchorPointY>
  </AnchorPoint>
  <Displacement>
    <DisplacementX>0</DisplacementX>
    <DisplacementY>25</DisplacementY>
  </Displacement>
  <Rotation>-45</Rotation>
</PointPlacement>
\end{lstlisting}

\textbf{Opacity:} Opacity is the transparency of either a label, point, polygon or line. It can be used to paint layers over each other (setting Opacity to 0), this is something used a lot incase multiple data has to be displayed in the same tile (used in our example as well). The way you use it is the following:
\begin{lstlisting}
<Opacity>0.3</Opacity>
\end{lstlisting}

\textbf{Rotation:} Rotation is the function that is used to turn all shapes and labels in SLD. It is very handy if you want to turn tiles or make labels line up with lines better. The way to use it is very simple in the section that has to be rotated just add the following code: \lstinline|<Rotation>-45</Rotation>| for a negative 45 degree turn.

\textbf{Graphic Fill:} A graphic fill is used in case a picture/image has to be shown in a layer. It has a lot of possibilities since every picture/image can be added through this way. The implementation is a little more complex so below is an example of a graphic fill. 
\begin{lstlisting}
<FeatureTypeStyle>
  <Rule>
    <PolygonSymbolizer>
      <Fill>
        <GraphicFill>
          <Graphic>
            <ExternalGraphic>
              <OnlineResource
              xlink:type="simple"
              xlink:href="colorblocks.png" />
              <Format>image/png</Format>
            </ExternalGraphic>
            <Size>93</Size>
          </Graphic>
        </GraphicFill>
      </Fill>
    </PolygonSymbolizer>
  </Rule>
</FeatureTypeStyle>
\end{lstlisting}
There are a lot more options and a lot of the information can be found in the SLD cookbook on the GeoServer website. This section was meant to give some more insight the commonly used functions.
  
