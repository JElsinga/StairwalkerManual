At the moment the stairwalker program only works with the PostgreSQL\footnote{PostgreSQL: \url{http://www.postgresql.org}} and MonetDB\footnote{MonteDB: \url{http://www.monetdb.org}} database management systems. This manual will only concern itself with PostgreSQL (shortened to Postgres). Any further reference to a database implies a Postgres database.

In this section a walk through explaining the steps needed to setup PostgreSQL along with PostGIS\footnote{PostGIS: \url{http://www.postgis.net}} and a custom made database extension for stairwalker on an Ubuntu Linux kernel.

\subsubsection{PostgreSQL Database Manager Setup}
Installing Postgres on Linux should be straight forward. In the terminal search for postgresql and then choose which version of postgresql should be installed. Below the commands used to search and install version 9.1 of Postgres are shown.
\begin{enumerate}
	\item \lstinline|aptitude search postgresql|
	\item \lstinline|apt-get install postgresql-9.1|
\end{enumerate}
For further information or if there are difficulties with the installation more details can be found on the installation page of the \href{http://www.postgresql.org/download/}{PostgreSQL website}.  

\subsubsection{PostgreSQL Configuration}
Once Postgres is installed on Linux two alterations will need to be made to the configuration files so that the database can be accessed from outside and by other users.

First off in the \lstinline|postgresql.conf| file the listen\_addresses need to be changed from localhost to all. This can be done as follows (assuming version 9.1 of postgres).

\begin{enumerate}
	\item \lstinline|cd /etc/postgresql/9.1/main|
	\item \lstinline|vi postgresql.conf|
	\item \textit{change} listen\_addresses = `localhost' $\rightarrow$ listen\_addresses = `*'
	\item \textit{save and close}
\end{enumerate}

\noindent Secondly in the file \lstinline|pg_hba.conf| a line needs to be added to allow other users in Linux to access postgres. This can be done as follows:

\begin{enumerate}[resume]
	\item \lstinline|vi pg_hba.conf|
	\item \textit{Add} host all	all 0.0.0.0/0 password
	\item \textit{save and close}
	\item \lstinline|/etc/init.d/postgresql restart|
\end{enumerate}

\noindent It should now be possible to create a database user and database in postgres. More information about how this is done can be found on the \href{http://www.postgresql.org/docs/manuals/}{Postgres manuals webpage}.

\subsubsection{PostGIS Configuration in PostgreSQL}
The next step is to extend Postgres with PostGIS database expansion. This again should be straight forward, first search for PostGIS and then choice the version that goes with Postgres to install. Note in order to do the install root privileges are required. Below the commands to search and install PostGIS for Postgres-9.1 are shown.

\begin{enumerate}
	\item \lstinline|aptitude search postgis|
	\item \lstinline|apt-get install postgresql-9.1-postgis|
\end{enumerate} 

\noindent More information about installing PostGIS  can be found on the \href{http://www.postgis.net/install}{PostGIS website}.
\newline
\newline
%I am not so sure why this is done or if it is even neccesary.
Once PostGIS is installed some extra configuration is required. This needs to be done as postgres user. The commands are as follows.

\todo[inline, size=\small]{Why do we do this again? What does it do?}
\begin{enumerate}	
	\item \lstinline|createlang plpgsql geonames|
	\item \lstinline|psql -f `find/usr/share/postgresql/ -name postgis.sql -print' -d geonames|
	\item \lstinline|psql -f `find/usr/share/postgresql/ -name spatial_ref_sys.sql -print' -d geonames|
	\item \lstinline|psql -f `find/usr/share/postgresql/ -name postgis_comments.sql -print' -d geonames|
\end{enumerate}

\subsubsection{Serverside Stairwalker Extension in PostgreSQL}
Finally there is C extension which allows processing of the SQL queries for the pre-aggregated database to happen on the server side instead of in the GeoServer application. This extension has to be installed on the Postgres database. The extension can be found in the directory \lstinline|neogeo/pre-aggregate/src/db-extensions/postgres/pa_grid|.

The extension can be installed on Linux using the following commands.

\begin{enumerate}
	\item \textit{go to the pa\_grid\ directory in the terminal}
	\item \lstinline|make| \\ \textit{this creates the dynamic library}
	\item \lstinline|make install| \\ \textit{this installs the library in the postgres installation}
	\item \lstinline|make sql| \\ \textit{declare the module in the database wanted}
\end{enumerate}

\noindent\textbf{Note} the extension has to be installed specifically for the database which will be used for pre-aggregation. In the makefile (also in the \lstinline|pa_grid| folder) there is a \lstinline|DATABASE| macro which should be set to the desired database.

\todo[inline, size=\small]{Installing the extension also requires root access. Why was this again?}
