Currently the Stairwalker program only works with the
PostgreSQL\footnote{PostgreSQL: \url{http://www.postgresql.org}} and
MonetDB\footnote{MonteDB: \url{http://www.monetdb.org}} database management
systems. This manual will only concern itself with PostgreSQL. Any further
reference to a database implies a PostgreSQL database.

In this section, we give a walk through explaining the steps needed to
setup PostgreSQL along with PostGIS\footnote{PostGIS:
\url{http://www.postgis.net}} and the database extension for
Stairwalker on an Ubuntu Linux operating system.

\subsubsection{PostgreSQL Database Manager Setup}
\label{sec:postgresql}

Installing Postgres on Linux should be straightforward. Search and install
the desired version of PostgreSQL. Below the commands used to search and
install version 9.1 of Postgres are shown.

\begin{enumerate}[noitemsep]
	\item \lstinline|aptitude search postgresql|
	\item \lstinline|apt-get install postgresql-9.1|
\end{enumerate}

For further information or if there are difficulties with the installation
more details can be found on the installation page of the
\href{http://www.postgresql.org/download/}{PostgreSQL website}.  

\subsubsection{PostgreSQL Configuration}

Once Postgres is installed on Linux two alterations will need to be made to
the configuration files so that the database can be accessed from outside
and by other users.

First, in the \lstinline|Postgresql.conf| file the
\lstinline|listen_addresses| need to be changed from \lstinline|localhost|
to \lstinline|all|. Assuming version 9.1 of Postgres this can be done as
follows.

\begin{enumerate}[noitemsep]
	\item \lstinline|cd /etc/postgresql/9.1/main|
	\item \lstinline|vi postgresql.conf|
	\item \textit{change} \lstinline|listen_addresses = `localhost'| \textit{to} \lstinline|listen_addresses = `*'|
	\item \textit{save and close}
\end{enumerate}

Secondly in the file \lstinline|pg_hba.conf| a line needs to be added to allow other users in Linux to access Postgres. This can be done as follows:

\begin{enumerate}[noitemsep,resume]
	\item \lstinline|vi pg_hba.conf|
	\item \textit{Add} \lstinline|host all all 0.0.0.0/0 password|
	\item \textit{save and close}
	\item \lstinline|/etc/init.d/postgresql restart|
\end{enumerate}

\noindent It should now be possible to create a database user and database
in PostgreSQL. In the sequel, we assume a database `\texttt{geonames}' is
created and used. More information about how this is done can be found on the
\href{http://www.postgresql.org/docs/manuals/}{\textsf{PostgreSQL manuals
webpage}}.

\subsubsection{PostGIS Configuration in PostgreSQL}
\label{sec:postgis}

The next step is to extend PostgreSQL with the PostGIS database expansion.
This again should be straightforward: first search for PostGIS and then
choose the version that goes with PostgreSQL to install. Note in order to do
the install root privileges are required. Below the commands to search and
install PostGIS for PostgreSQL-9.1 are shown.

\begin{enumerate}[noitemsep]
	\item \lstinline|aptitude search postgis|
	\item \lstinline|apt-get install postgresql-9.1-postgis|
\end{enumerate} 

\noindent More information about installing PostGIS can be found on the
\href{http://www.postgis.net/install}{\textsf{PostGIS website}}.

Once PostGIS is installed some extra configuration is required to add
PostGIS functionality to the used database (in our case
`\texttt{geonames}). This needs to be done as postgres user. The commands
are as follows.
\begin{enumerate}[noitemsep]
	\item \lstinline|createlang plpgsql geonames|
	\item \lstinline|psql -f `find/usr/share/postgresql/ -name postgis.sql -print' -d geonames|
	\item \lstinline|psql -f `find/usr/share/postgresql/ -name spatial_ref_sys.sql -print' -d geonames|
	\item \lstinline|psql -f `find/usr/share/postgresql/ -name postgis_comments.sql -print' -d geonames|
\end{enumerate}

\subsubsection{Serverside Stairwalker Extension in PostgreSQL}
\label{sec:serversideextension}

Stairwalker is a database extension written in C.  The extension has to be
compiled and installed in the PostgreSQL database. The extension can be
found in the directory
\lstinline|neogeo/pre-aggregate/src/db-extensions/postgres/pa_grid|.

The extension can be installed on Linux using the following commands.
\begin{enumerate}[noitemsep]
	\item \textit{go to the pa\_grid\ directory}
	\item \lstinline|make| \\ \textit{this creates the dynamic library}
	\item \lstinline|make install| \\ \textit{this installs the library in
		the PostgreSQL installation}
	\item \lstinline|make sql| \\ \textit{declare the module in the desired
		database}
\end{enumerate}

\noindent\textbf{Note} the extension has to be installed specifically for
the database which will be used for pre-aggregation. In the makefile (also
in the \lstinline|pa_grid| folder) there is a \lstinline|DATABASE| macro
which should be set to the desired database.

%\todo[inline, size=\small]{Installing the extension also requires root access. Why was this again?}
