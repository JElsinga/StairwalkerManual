%Now in order to see the layer go to view, here it should display the layer that has been made (with the added name) with a drop-down menu. Select PNG/JPEG or something similar to show a basecase of the data. A new window should open and here it should show your data in a weird graph. If this is not the case, its likely a mistake in the style. It is also possible to adjust your view in the images by adding a VIEWPARAM to the url. To do so just add: \&VIEWPARAMS=$<$Type$>$:$<$Value$>$ to the url and the image should change.

In GeoServer it is possible to get a preview of layer such as the one created in \Fref{sec:addinglayers}. Previewing a layer can be done as follows:

\begin{enumerate}
	\item Navigate to \lstinline|Layer Preview| by clicking \lstinline|Layer Preview| link under the \lstinline|Data| section in the navigator on the left hand side of the web administration interface homepage.
	\item The \lstinline|Layer Preview| page will have a list of all configured layers with can be previewed in various formats.
	\item Locate the layer which should be shown and from the \lstinline|All Formats| column choice any \lstinline|WMS| format.
	\item After selecting a format to view the layer a new page will open with a visual representation of the top most layer of dataset.
\end{enumerate}

Note that other preview formats should also be possible. For example it is possible to use the OpenLayers preview format which allows one to navigate the geospatial data. In section \Fref{sec:clientsidedev} OpenLayers will also be used to visualize the dataset on a web page. One drawback is that for every movement in the preview a new query has to be calculated. When server side stairwalker (\Fref{sec:serversideextension}) is not setup calculating can be time consuming. Therefore during development of the pre-aggregation index and testing it is advices to use a static preview and only once everything works as desired to use a dynamic preview.

If the layer contains a nominal axis it is possible to alter the value of the nominal with which the data is filtered. This is done by adding a parameter in the request made to the GeoServer extension. By extending the \lstinline|HTML| request sent to GeoServer with \lstinline|&VIEWPARAMS=<TYPE>:<VALUE>;| the nominal axis filter is used. Currently the extension only supports the nominal type \lstinline|keyword|. If the pre-aggregate index was created using a nominal axis (splitting on \lstinline|VALUE|s) then using \lstinline|&VIEWPARAMS| will split the visualization on a give \lstinline|VALUE|. \lstinline|&VIEWPARAMS| can accept more \lstinline|<TYPE>:<VALUE>;| tuples, however parsing type will need to be extended on in the code. For more information see \Fref{sec:filtering}.

It may be that a preview fails to load, this can be due to two reasons. The first is an error in the style, in this case the style needs to be tested which can be done by validating the style in GeoServer. For more information about styles see \Fref{sec:visualization}. The second reason is an error is creating an SQL query for the pre-aggregated index. If no mistakes where made during setup and in creating a pre-aggregate index there will be a need to dive into the code where detailed logging is done.
\todo[inline, size=\small]{This would be a good point to reference the logging done in the code however i cannot find where this log file is located, debugging is also possible using the console in an IDE.}

