In order to add a source to Geoserver you first need to be logged into the Geoserver webapplication. \url{http://localhost:8080/geoserver/web} is the link to the webapplication of Geoserver. The standard login and password for this are:
\newline
\newline
Username: admin
\noindent Password: geoserver
\newline
\newline
Once the webapplication has logged in, then start by adding the source of the database. 
\begin{enumerate}
	\item On the left side there should be a menu with all kinds of options, but in order to add a source, go to Configuration $\rightarrow$ Sources. 
	\item At the top there should be an add source button. 
	\item Now under vectortypes there should be a list of types. Select what kind of source you want to add. In order to get the Stairwalker program running there should be a NeoGeo Aggregate - NeoGeo aggregation index query. If this is not the case something went wrong with deploying the geoserver and the last steps in this manual should be done again. 
	\item If it is here, click on it and a new menu will open. A new screen should open where all information for your source has to be filled in. There is also an option on what aggregation type the program should use, this should have been defined before so use what has been used before (count/sum/minimum/maximum). 
	\item Once everything is filled in all, the source should have been added to the Geoserver. This can be checked by going to sources again and then it should display the source that has just been added.
\end{enumerate}