The next step is to make a style in which you want your data to be shown. A style has to be made in SLD/XML, there are a lot of options which can be used to make a style. First navigate to the style tab, Configuration $\rightarrow$ Style here you can add a new style with the button on top. Here you need to add your code (or file) for your style.
\newline
\newline http://docs.geoserver.org/2.5.x/en/user/styling/sld-cookbook/index.html contains a lot of information regarding styles, it is a lot of trial and error to get a nice working style. The next few steps however should give a short setup on how to start making your own style.
\begin{enumerate}
	\item Decide what data should be shown, in our example we’re using the count of the aggregated data.
	\item Decide what kind of representation should be used, you can either use Polygonsymbolizer, pointsymbolizer or linesymbolizer. All of these have their own advantages and disadvantages, incase it isn't clear what to use, start off with a Polygonsymbolizer.
	\item Add all filters, for instance if the data should how a lighter color if the amount is lower. Or make a tile red once the data exceed a certain maximum (for instance if there were too many orders in a timerange).
	\item Add colors to the layers representing it better and giving it a nice layout.
	\item Make sure the positioning of the label and all filters are working.
	\item Finish the layer by adding the last details (Halo behind the text for example).
\end{enumerate}
	This is just a basic manual on how to add a style, for more information or more explanation on how to make a style, check out the development chapter in this manual. In that section there are far more details on how to make a style.